\documentclass[a4paper, 12pt]{article}
\begin{document}

\title{Exapmle}
\author{Patrick}
\date{Dec 2, 2020}
\maketitle

\section{What's this?}
This is our second document. It contains a title and a section with text.

It contains two paragraphs. The first line of a paragraph will be indented, but not when it follows a heading.
% Here's a comment.

\section{Try symbols}
Statement \#1:
50\% of \$100 makes \$50.
More special symbols are \&, \_, \{ and \}.\\
``\textbackslash textbackslash" is the command for printing \textbackslash.\\
``\textbackslash\textbackslash" stands for line breaking command.

\section{Formatting text}
Text can be \emph{emphasized}.

Besides being \textit{italic} words could be \textbf{bold}, \textsl{slanted} or
  typeset in \textsc{Small Caps}.

Such commands can be \textit{\textbf{nested}}.

\emph{See how \emph{emphasizing} looks whe nested}

\section{\textsf{\LaTeX\ resources on the internet}}
The best place for downloading LaTeX related software is CTAN. \\
Its address is \texttt{http://www.ctan.org}.

% Switching the font familly
\section{\sffamily \LaTeX\ resources in the internet}
Tthe best place for downloading \LaTeX\ related software is CTAN. \\
Its address is \ttfamily http://www.ctan.org\rmfamily.

\textbf{Summarizing font commands and declarations}

\begin{center}
  \begin{tabular}{ |c c c c| }
    \hline
    \head{Command} & \head{Declaration} & \head{meaning} & \head{Output} \\
    \verb|\textbackslash textrm{...}| \verb|\textbackslash rmfamily| & \verb|roman family| & \rmfamily Example text\\
    \hline
  \end{tabular}

\end{center}
\end{document}
